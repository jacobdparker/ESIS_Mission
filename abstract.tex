The Extreme ultraviolet Snapshot Imaging Spectrograph (ESIS) launched on a sounding rocket from White Sands Missile Range on September 30, 2019.
ESIS is a Computed Tomography Imaging Spectrograph (CTIS) designed to capture both spectral and spatial information simultaneously over a large field of view to provide velocity information on small scale transient events that are prevalent at transition region temperatures.
In this paper, we review the ESIS instrument, mission, and the data captured.
We demonstrate how this unique data set can be interpreted qualitatively, and further present some initial quantitative inversions of the data.
Using a Multiplicative Algebraic Reconstruction Technique (MART) we combine information from all four ESIS channels into a single spatial-spectral cube at every exposure.
We analyze two small explosive events in the \ov \ spectral line with jets near $\pm 100$\,km/s that evolve on 10\,s time scales and vary significantly over small spatial scales. Intriguingly, each of these events turns out to be a bimodal (red+blue) jet with outflows that are asymmetric and unsynchronized.
We also present a qualitative analysis of a small jet-like eruption captured by ESIS, and draw early comparisons to previously observed mini-filament eruptions.
In the 5 minutes of observing time, ESIS captured the spatial and temporal evolution of tens of these small events across the $\approx 11.5'$ field of view, as well as several larger extended eruptions, demonstrating the advantage of CTIS instruments over traditional slit spectrographs in capturing the spatial and spectral information of dynamic solar features across large fields of view.
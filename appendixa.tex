We developed a new image despiking algorithm with a spike intensity threshold that varies according to the local median intensity. The key to our strategy is to establish a finite number of bins for the local median intensity, and then examine the statistics of all the pixels falling into each bin. Our despiking routine is implemented in the following steps:
\begin{enumerate}
	\item For each data dimension (image row, image column, and time), calculate a local median intensity image by averaging the nearest 11 pixels along the given axis. Divide the range of median image values into 128 bins.
	\item For each local median bin, form a cumulative distribution of intensities. Identify the intensity value corresponding the 99.9th percentile. The result is three curves (one for each data dimension) of 128 points each. 
	\item For each data dimension, fit a line to the threshold curve. The line identifies the relationship between median intensity and spike threshold for that dimension.
	\item Create three masks, which are  boolean-valued images identifying pixel values that exceed the thresholds, one image per data dimension.
	\item Combine the three masks with logical AND to produce the final map of bad pixels (spikes). 
	\item Replace each bad pixel with a local mean of good (i.e., not bad) pixels only, weighted by kernel $k$ (Equation \ref{eq:despike_kernel}).
\end{enumerate}
The kernel used to determine the replacement value for bad pixels is
\begin{equation} \label{eq:despike_kernel}
k(x,y,t) = e^{-\lbrace|x|+|y|+|t|\rbrace/a}.
\end{equation}
For our implementation $a=.5$\,pixels and $x,y,t= -2, -1, 0, 1 , 2$.
The kernel is normalized prior to application by it's total, ignoring any other bad pixels that may fall within the kernel.

The algorithm described above has a number of advantages compared to those available in standard data prep routines for solar physics missions. 
We are aware of only one published example of a despiker that uses the time axis for spike identification, used by \citet{Aschwanden2000(trace_unspike_time)} on data from the Transition Region and Coronal Explorer \citep[TRACE]{handy1999}, \edit1{though other sophisticated despikers have been used in processing data from SECHI/STEREO \citep{Howard2008} and MDI \citep{DeForest2007}.}

In our algorithm, a pixel must stand out from its neighbors in all three dimensions to be identified as a spike.
This criterion reduces false positives; yet because of this approach, we can choose a relatively sensitive percentile threshold based on image statistics to reduce false negatives. 
Usually, a percentile-based threshold would result in identifying a fixed number of bad pixels; but since the logical AND is nonlinear with respect to the three binary masks, it is possible to identify a small or large number of spikes in a given image. 
For example, with our 99.9th percentile threshold, the algorithm can find as few as zero or as many as 2129 bad pixels in a $2048\times 1040$ Level 1 image.


\chapter{First Flight of the EUV Snapshot Imaging Spectrograph (ESIS)}

\section{Abstract}

\section{Introduction}
    \begin{itemize}
        \item Instrument Heritage
            Brief summary since this is a repeat of the instrument paper.
        \item Scientific Motivation and Goals
            There is ALOT of this copied from the propsoal into the instrument paper.  I think it needs to be lightly summarized here.
        \item Sections Outline
    \end{itemize}
    

\section{ESIS Mission}
    \begin{itemize}
        \item Brief description of the instrument, mostly pointing to the instrument paper.  Describe ESIS enough so that the data levels make sense. Need to consult the instrument paper to see what fits here. 
        \item ESIS Flight info.  Time, airtime, pointing, stability, etc.  Suggestions welcome here.
        \item Summary of Coordinated Data????
    \end{itemize}
	
\section{Data}

    \subsection{Level 1 Data}
        \begin{itemize}
            \item Bias Subtraction
            \item Gain Correction
         
            
                 Get in touch with MSFC about different apparent channel gains.
            \item Dark Subtraction
            \item Despiking
        \end{itemize}
       
	       


    \subsection{Level 3 Data}
        \begin{itemize}
            \item Co-Alignment to AIA 304
            \item Quadratic Internal Alignment
            \item Vignetting Correction
            \item Conversion to Photons
        \end{itemize}
        
        The ESIS Level 3 data product was created to provide a co-aligned, single wavelength image in each channel for quicker identification of events with non-zero LOS velocities, and easier comparison with coordinated data that don't require inversion. 
        It will also allow for single wavelength inversion with assumptions prior to the completion of a more complete optical distortion model. 
        A quick method for identifying solar events with significant velocity is by taking the difference between two ESIS channels.
        Each ESIS channel disperses features with non-zero LOS velocities a different direction, determined by the asymuthal position of each grating. 
        Stationary solar features will have the same presentation in each camera and moving features will be smeared in different directions.
        Therefore, significant differences between ESIS images, for a given wavelength, highlight bright solar features with significant velocity.
        
        The four ESIS channels were spatially co-aligned in two steps.  
        First, each ESIS image is cropped around the desired spectral and then co-aligned to the closest AIA 304 image in time.  
        AIA 304 was chosen because, despite a formation temperature difference, it is the AIA EUV channel most visually similar to O V, in background and bright events.
        The co-alignment was achieved through a linear transformation of the cropped ESIS image that maximizes the zero lag cross-correlation between it and AIA 304.
        Despite each image being in a different wavelengths we found normalized peak zero lag cross-correlations around .9 (\textbf{PULLING THIS NUMBER OUT OF MY ASS FOR NOW, BUT IT'S CLOSE.})
        After this step each ESIS channel has been re-binned to AIA resolution and can be assigned WCS information based on AIA Level 1.5 (\textbf{ADD NOTES ABOUT STANDARD AIA PREP Above)}.
        Since ESIS has a slightly non-linear distortion function \citep{ESIS}, an additional internal alignment step is performed.
        Choosing a single ESIS channel as reference, in our case Camera 1, each other camera is co-aligned to it via a quadratic transformation that maximizes the zero-lag cross-correlation between it and the reference channel.
        After performing this additional internal alignment we find that not only is the zero-lag cross-correlation between each channel and the reference channel improved, but also the cross correlation between each combination of channels.
        
        \textbf{Time for a Figure.}
        
        In order to use Level 3 data for early inversion the intensity needs to normalized between channels and converted to photons to properly account for uncertainty.
        Since each ESIS CCD has a quadrant specific gain \citep{ESIS}, the conversion from DN to photon is done to Level 1 data prior to co-alignment efforts.
        The wavelength used when converting to photons corresponds to the spectral line cropped out of Level 1 prior to co-alignment.
        Each ESIS channel has a linear trend in intensity along the dispersion direction due to internal vignetting in the optical system \citep{ESIS}.
        This linear trend in the background is apparent in difference images that haven't been corrected.
        
        \textbf{Figure, Show Corrected and Uncorrected?}
        
        To correct the trend, we divide out a linear trending background from each image with a slope oriented to the dispersion direction in each channel, each with a different slope.
        The slope of each background is a free parameter in our fit.
        For each of the four ESIS channels we mask off the portion common to them all with no contribution from Mg X (\textbf{wavelength?}).
        The four channels of ESIS give 6 possible difference images for fitting the vignetting function. 
        For each difference image we take a mean of each column and fit a line to it as a function of column position.
        When the average slope of all six fits is minimized we consider the vignetting corrected. 
        
\section{Events}
    \subsection{Explosive Events}
        \begin{itemize}
            \item Quantity, Spatial and Temporal Scale
            \item Burstiness 
        \end{itemize}
    
    \subsection{Main Event}

\section{Discussion/Conclusions and Future Work}


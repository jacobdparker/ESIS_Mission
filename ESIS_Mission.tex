\renewcommand\arcmin{\mbox{$^\prime$}}
\renewcommand\arcsec{\mbox{$^{\prime\prime}$}}
% Above taken straight from aastex63.cls, needed for use with other cls files

\newcommand{\rts}[1]{{\color{violet} RTS: #1}} % RTS comment
\newcommand{\jdp}[1]{{\color{red} JDP: #1}} % JDP comment
\newcommand{\cck}[1]{{\color{brown} CCK: #1}} % CCK comment
\newcommand{\amy}[1]{{\color{cyan} ARW: #1}}

% #1- element #2- ionization state #3 -wavelength in angstroms
\newcommand{\spectralline}[3]{#1\,{\textsc{#2}}\ #3\,\AA } % cck fixed up typesetting of this command....
\newcommand{\ov}{\spectralline{O}{v}{629.7}}
\newcommand{\oiii}{\spectralline{O}{iii}{599.6}}
\newcommand{\oiv}{\spectralline{O}{iv}{608.4}}
\newcommand{\oivTwo}{\spectralline{O}{iv}{609.83}}
\newcommand{\mgxbright}{\spectralline{Mg}{x}{609.79}}
\newcommand{\mgxdim}{\spectralline{Mg}{x}{624.9}}
\newcommand{\hei}{\spectralline{He}{i}{584.3}}
\newcommand{\heii}{\spectralline{He}{ii}{304}}
\newcommand{\siiv}{\spectralline{Si}{iv}{1394}}

%ESIS pointing info
\newcommand{\esispointing}{[18\arcsec, -19\arcsec]}
\newcommand{\esisroll}{\ensuremath{0.85^{\circ}}}
\newcommand{\esisfov}{11.5\arcmin}
\newcommand{\aianearapogee}{18:08:17\,UTC}





% This gives us \sout for striking out text in comments....
% \sout command defined in aastex63.cls

\title{First Flight of the EUV Snapshot Imaging Spectrograph (ESIS)}

\author[0000-0001-8732-8284]{Jacob D. Parker}
\affiliation{Montana State University}
\author[0000-0002-9997-5515]{Roy Smart}
\affiliation{Montana State University}
\author[0000-0002-1992-7469]{Charles Kankelborg}
\affiliation{Montana State University}
\author[0000-0002-5608-531X]{Amy Winebarger}
\affiliation{NASA Marshall Space Flight Center}
\author[0000-0002-6787-0380]{Nelson Goldsworth}
\affiliation{Montana State University}

\begin{abstract}
    The Extreme ultraviolet Snapshot Imaging Spectrograph (ESIS) was launched on a sounding rocket from White Sands Missile Range on September 30, 2019.
    ESIS is a Computed Tomography Imaging Spectrograph (CTIS) designed to capture both spectral and spatial information simultaneously over a large field of view to provide velocity information on small scale transient events that are prevalent at transition region temperatures.
    In this paper, we review the ESIS instrument, mission, and the data captured.
    We demonstrate how this unique data set can be interpreted qualitatively, and further present some initial quantitative inversions of the data.
    Using a Multiplicative Algebraic Reconstruction Technique (MART) we are able to combine information from all four ESIS channels into a single spatial-spectral cube at every exposure.
    We analyze two small explosive events in the \ov \ spectral line with jets  near $\pm 100$\,km/s that evolve on 10\,s time scales and vary significantly over small spatial scales.
    In the 5 minutes of observing time, ESIS captured tens of the small events across the $\approx 11.5'$ field of view, as well as several larger extended eruptions, demonstrating the advantage of CTIS instruments over traditional slit spectrographs in capturing the spatial and spectral information of dynamic solar features across large fields of view. \cck{Once we are settled on the conclusions section, let's review the Abstract to see if more science can be added.}
  	
\end{abstract} 

\section{Introduction}

    Observationally, the solar transition region (TR) refers to the portion of the solar atmosphere at temperatures between 20,000\,K and 800,000\,K. 
    Initially, the transition region was viewed as simply the thin interface region between the dense, cool chromosphere and tenuous, million-degree corona, where the temperature of the plasma dramatically increased two orders of magnitude over tens of kilometers \citep[see][and references therein]{tian2017}. 
    Though undoubtedly this type of transition region exists in hot coronal loops, the concept of the transition region has been expanded over the last twenty years to include a dynamic and complicated three-dimensional geometry. 
    %It is in the low transition region where the magnetic pressure begins dominate the plasma pressure. 
    
    The transition region is rife with magnetically driven phenomena such as explosive events \cite[e.g.,][]{dere1991} and microflares \citep{gontikakis2012} which manifest supersonic flows.
    Explosive events were discovered in rocket flights \citep{Dere1989}. 
    Explosive event outflows are typically of order $\pm 100$\,km/s. 
    The release of free magnetic energy in the low-$\beta$ plasma was implicated because these flows are much faster than the TR sound speed (e.g., $\approx55$\,km/s  for C\,\textsc{iv}, formed at $10^5$\,K).
    Explosive events have been found to present in different ways and have been observed by many instruments since their original discovery.
    \citet{innes1997} found fast bi-directional jets in \siiv \ line profiles taken by the Solar Ultraviolet Measurements of Emitted Radiation instrument (SUMER)  \citep{SUMER} separated by a few arcseconds and inferred they must be associated with the outflows of an inclined reconnection current sheet.
    Higher resolution observations from the Interface Region Imaging Spectrograph \citep[IRIS]{depontieu2014} have shown that Si\,{\sc iv} line profiles often show very bright line cores with broad wings and very non-gaussian, more triangular profiles.% \amy{add ref?}\jdp{the next reference is intended to back this up.  Do we need more?}
    This enhanced line core emission points to a larger amount of stationary plasma, and has been attributed to the presence of a dominate Tearing Mode Instability during the reconnection \citep{Innes2015}.
    Enhanced emission in the line core of a larger explosive event viewed in \heii \ by the Multi-Order Solar Extereme-ultraviolet Spectrograph (MOSES)  has also been attributed to the tearing mode instability \citep{Fox10} despite tens of smaller events in the same data having clear bi-modal profiles \citep{Rust2019}.
    In order to determine which, if any, of these presentations is typical we require high cadence velocity data over a wide range of temperatures simultaneously, something difficult to achieve even in multi-instrument studies.
    
    
    Investigations of TR events to date are severely limited by available observational capabilities. 
    Historically, slit spectrographs have been used to determine velocity of the plasma in the solar atmosphere.   
    Spectrally resolved images of the sky plane must be built by rastering the slit over the region of interest, which takes much longer than the timescales of TR phenomena and confuses whether events are evolving temporally or varying spatially.  
    
    One solution to simultaneously capture spatial and spectral information over a large field of view is to use a slitless imaging spectrograph.  
    The data from these instruments, sometimes called overlapograms, have spatial and spectral information overlapped in the dispersive direction, requiring the data to be inverted or unfolded.
    The difficulty in unfolding slitless spectrograph data has limited its usefulness for extended astrophysical objects like the Sun. 
    Only two satellite missions have routinely captured solar slitless spectrograph data: The S082A instrument on {\it Skylab} \citep{Tousey1973} and the Res-K instrument of the Russian KRONOS-I mission \citep{Zhitnik1998}, though others have been recently developed and proposed \citep{winebarger2019,golub2020}. 
    Additionally, the currently-operating Extreme-ultraviolet Imaging Spectrograph (EIS; \citet{culhane2007}) on the {\it Hinode} mission \citep{kosugi2007} includes 40\arcsec\ and 266\arcsec\ slots that can produce overlappogram data.  
    Though EIS slot data are not often studied quantitatively, % taken for scientific analysis, 
    they have been used as a flare trigger and since analyzed to aid in interpretation of impulsive phase of solar flares \citep{harra2017,harra2020}.
    In addition to these satellite observatories, %there have been several solar observations with slitless spectrographs on sounding rocket flights, including 
    the Multi-Order Solar EUV Spectrograph (MOSES) instrument by Kankelborg and collaborators \citep{Kankelborg01,Fox10,Rust2019} has observed dynamic events in the solar transition region.
    MOSES captured the zero and plus and minus one order of the \heii \ line simultaneously. 
    Doppler shifts were then detected as the spectral displacements in opposite directions in the $\pm$ 1 orders.
    
    Slitless spectrograph data can be thought of as a projection of a three dimensional spatial-spectral data cube, $I(x,y,\lambda)$, onto a two dimensional detector.  
    Although {\it Skylab} S082A had just a single projection through $x,y,\lambda$-space, it was possible under some circumstances to determine line intensities and line ratios \cite[e.g.,][]{Keenan1988, Tayal1989, Keenan2006}, and even Doppler shifts \citep{MariskaDoppler1992}. 
%    \jdp{Maybe in O V even!} Nope, Ne VII
    Slitless spectrographs using two projections (usually one dispersed, and one undispersed) have proven sufficient to implement an efficient ground-based magnetograph \citep{DeforestStereoscopy2004}, map Doppler shifts in the solar transition region \citep{Courrier2018}, and invert temperature or density information \citep{winebarger2019}. 
    However, as with any tomographic inversion problem \citep[e.g.,][]{KakSlaney2001}, the fidelity of reconstruction improves dramatically as projections are added. 
    This is particularly true as the complexity of the object increases, causing the overlap of multiple features along the projection. 
    An instrument that captures multiple projections of the spatial-spectral data cube is called a Computed Tomography Imaging Spectrograph  \cite[CTIS,][]{DescourDereniakCTIS1995}.  
    MOSES is an example of a CTIS, as it captured three projections in the $\pm$ 1 and 0 orders.  
    \cite{Fox10} was able to extract line widths and doppler shifts from a relatively complex explosive event observed by MOSES in \heii.  
    \cite{Rust2019} found unambiguous evidence of explosive events with well resolved, double-peaked spectral line profiles in the same data set.
    
    In 2013, the Extreme-ultraviolet Snapshot Imaging Spectrograph (ESIS) was proposed to the NASA Low Cost Access to Space (LCAS) program and subsequently selected.  
    ESIS is a CTIS with four unique projections of the spatial-spectral data cube and is designed to capture velocity information in small-scale reconnection events in the \ov \ emission line that is formed in the solar transition region.  
%    ESIS was designed to be flown on the opposite side of the MOSES optics bench with the intention of both instruments gathering data simultaneously.   
    The ESIS mission was launched from White Sands Missile Range on September 30,  2019; this paper is a description of the ESIS mission, data and preliminary results.  
 %   Unfortunately MOSES was  not operational during the flight  and will not be discussed further.  
    The ESIS experiment, target, and flight is described in Section 2.  
    Section 3 provides information on the data processing.  Preliminary results are given in Section 4; these include both qualitative and quantitative measures of small-scale velocity events in the solar transition region.  The ESIS mission was successful in  observing tens of small-scale reconnection events over the short rocket flight, as well as demonstrating the usefulness of CTIS observations and developing the analysis techniques required to interpret this unique data.


\section{ESIS Mission}

In this section we provide an overview of the ESIS experiment as well as the time and conditions of the ESIS rocket launch and subsequent data collection.   

	\subsection{The Experiment}
	
		\begin{deluxetable}{l|r|r}
			\caption{Dominant spectral lines observed by ESIS.  Intensities are relative to \ov.}
			\label{tab:linelist}
			\tablehead{\colhead{Ion} & \colhead{$\lambda$ (\AA)} & \colhead{Intensity}}
			\startdata
			He\,{\sc i} & 584.33 & 0.70 \\
O\,{\sc iii} & 599.59 & 0.13 \\
O\,{\sc iv} & 608.40 & 0.06 \\
Mg\,{\sc x} & 609.79 & 0.25 \\
O\,{\sc iv} & 609.83 & 0.11 \\
Mg\,{\sc x} & 624.94 & 0.13 \\
O\,{\sc v} & 629.73 & 1.00

			\enddata
		\end{deluxetable}
	  	
    	The full ESIS experiment includes an optical instrument, a set of detectors, an on board data acquisition system, and ground support equipment and is described in great detail in the preceding paper \citep{ESIS}.
    	The ESIS optical design consists of a single parabolic primary mirror, an octagonal field stop placed at an intermediate focal plane, and 4 spherical diffraction gratings each with their own corresponding CCD detector.
    	Incoming light is focused by the primary mirror onto the octagonal field stop. 
    	The octagonal field stop is $\approx$ 5 mm wide, which is equivalent to % roughly 
    	11.5\arcmin \  when projected onto the sky. 
    	The light that exits the field stop is reimaged  by the four spherical diffraction gratings onto their own CCD.
    % 	operated in frame-transfer mode. Details like this are in the instrument paper 
    	The portion of the solar spectrum that is captured by each detector ranges from $\approx$ 584\,\AA \ to 630\,\AA. 
%    	Within this passband the dominant spectral lines are \hei, \oiii, \oiv, \mgxbright, \mgxdim, and \ov.
    	Table \ref{tab:linelist} shows the dominant lines observed by ESIS.
    	Intensities relative to \ov \ are calculated using Chianti \citep{ChiantiI,ChiantiX} assuming a constant log pressure of 15\,K\,cm$^{-3}$ and a quiet sun DEM, combined with the Schmeltz 2012 abundance file \citep{schmelz2012}.
   		While this paper focuses primariy on the dominant \ov \ spectral line, future work will be focused on measuring plasma velocity in other bright lines.
   		
    	

%    	 \ of 0.7, 0.124, 0.06, 0.36, 0.12, and 1 respectively.
%    	The 0.36 relative intensity of the \mgxbright \ line includes a blend with %contribution from 
%    	the \oivTwo \ line within 0.04\,\AA\ which, in this example, accounts for roughly one third of the intensity.
    	
     % at least the three dominant spectral lines, \hei, \mgxbright, and \ov.

    	
        \begin{figure}[ht]
			\begin{center}
				\includegraphics{detector_layout}
				\caption{A modeled layout showing the orientation of each ESIS channels' detector relative to the solar image at the octagonal fieldstop.  The locations of the brightest wavelengths (\hei, \mgxbright, and \ov) in the passband are shown on each detector.  The \hei \ line  extends off the detector edge by design, as can be seen in Figure \ref{fig:L0_to_L1}. }
				\label{fig:level_1_array}
			\end{center}
		\end{figure}

    	Each grating and detector pair, referred to as a channel in this paper,
    	is itself an independent slitless imaging spectrograph.  
    	The channels are arrayed at $45^{\circ}$ increments about the axis of symmetry of the paraboloidal primary mirror such that each channel disperses the solar image a different direction. 
    	Hence, each of these channels capture a unique projection of the spatial spectral cube, shown in Figure \ref{fig:level_1_array}. 
    	As shown, each channel captures the Sun imaged through the octagon and dispersed at different relative angles with respect to solar north. Combining these four channels creates a CTIS. 
    	Exposures from all four channels are gathered nearly simultaneously by triggering the frame transfer in three of the cameras by a single ``master'' camera. 
    	The ESIS instrument currently has 4 channels, but is built to accommodate up to 6 (limited by interference with the optical bench).

    	
        % The dark current in the ESIS cameras is reduced by cooling detectors to low temperatures. 
        % The CCDs in the ESIS cameras are mounted in a copper carrier, which is connected via a copper strap to a cold block.  
        % The cold block is cooled prior during testing and prior to flight to -120 $^{\circ}$C.  This cold block acts as a thermal reservoir to maintain the CCD temperature during the $\sim$ 10 minute rocket flight.   
        % ESIS exposures are transferred via spacewire to an on-board data acquisition system. During ground testing, including alignment, cameras can be commanded individually or collectively and images displayed via ethernet connection to an Operational Ground Support Equipment (GSE) computer.  
        % During flight, ESIS is designed to operate autonomously, though commands can be uplinked if required.  
        % Exposures are buffered, downlinked, and displayed in near real time on the Operational GSE.   These images can be used to make pointing corrections during flight if needed. \amy{This may be too much information, but I wanted to beef up this section.  
        % Also, wanted to highlight  that ESIS isn't just an optical system, it is optical + on board electronics + ground support equipment.}
	
    
	\subsection{Launch and Data Collection} 
		\begin{figure}[ht]
			\begin{center}
				\includegraphics{figures/signal_and_altitude_vs_time}
				\caption{The altitude of the ESIS rocket determined from White Sands Missile Range radar data as a function of elapsed time from launch.  The event times listed in Table~\ref{tab:data_info} are labeled.}
				\label{fig:timeline}
			\end{center}
		\end{figure}

		ESIS was launched at \timeMissionStart~UTC
		on \dateMission\ from White Sands Missile Range.  Figure~\ref{fig:timeline} provides the height of the sounding rocket as a function of time, determined from % White Sands Missile Range 
		radar measurements, as well as several key events during the flight.  
		The ESIS cameras began exposing at launch and continued to record full detector ($\sim$2k$\times$1k) images with a 10\,s exposure and cadence throughout the flight. Images taken while the
		experiment shutter door was closed (i.e., during the initial  boost phase of the rocket flight, and ballistic ascent to approximately 100\,km altitude) serve as darks.  
		At \timeMissionShutterOpen\ after launch, the shutter door %to the experiment 
		opened.  
		The Sun was acquired by the Solar Pointing And Rocket %and Aerobee 
		Control System (SPARCS) and the ring laser gyro (RLG) was enabled at \timeMissionRlgEnable\ indicating the rocket was in Fine Pointing Mode.  
		During this mode, SPARCS maintained a constant target; however, thermal expansion of optical components inside the instrument caused the apparent drift of the solar image on the detectors, as described in Section 3.  
		At  \timeMissionShutterClose, the shutter door closed ending solar observations. Exposures continued until the system shutdown at \timeDataStop, providing several additional dark frames at the end of the flight.   A summary of the flight and data collected is given in Table~\ref{tab:data_info}.
		
	    \dateMission\ was a very quiet day on the Sun.  
	    In fact, the last  B-class event detected by GOES \citep{GOES} prior to the ESIS launch was July 7, 2019.  Because of this exceptionally quiet period on the Sun, we chose to point at disk center. 
	    The actual pointing of the center of the ESIS octagonal field stop and roll were found after flight by comparing the ESIS exposure nearest apogee at \timeApogeeFrame\,UTC to the closest AIA\,304\,\AA\ image taken at \aianearapogee.  
	    The ESIS field of view projected onto the AIA 304\,\AA\ image is shown in Figure \ref{fig:fov} as a green octagon.  
	    The pointing was found to be $\approx$ \esispointing \ and the roll offset was found to be $\approx \esisroll$ \ (clockwise about Sun center), both within the tolerances for SPARCS pointing.  
	    Figure \ref{fig:fov} shows the full-disk AIA\,304\,\AA\ image. 
	    The ESIS FOV is indicated by the octagon.  
	    
		Each of the ESIS cameras collected 71 images during flight.  
		The first two images are unusable; the first image has an exposure time of 0\,s and is basically a dump of the dark current in the camera, while the second image has an exposure time slightly longer than 10\,s caused by the cameras syncing to a single trigger of the master camera.  
		All the other images have a 10\,s exposure time. 
		29 of these images were light frames, meaning the shutter door was open and the RLG was enabled. 
		The time during the rocket flight when light frames were collected is shaded green in Figure \ref{fig:timeline}.  
		Light emitted by the Sun is absorbed by the Earth's atmosphere.  
		The degree of absorption depends upon the column of the atmosphere that the telescope is looking through. The wavelength dependence of atmospheric absorption is weak, on the order of a few percent across the ESIS passband.  
		Figure~\ref{fig:timeline} shows the mean intensity in each of the ESIS channels as a function of time.  Atmospheric absorption clearly impacted the observations in the up and down legs of the flight.  
		Accounting for atmospheric absorption will be discussed in Section 3.  
		Though there are \numDataFrames\ light frames, the first and last several frames were greatly impacted by atmospheric absorption and have limited usefulness.  As the rocket was re-entering the atmosphere, a transient signal affects some of the images that would otherwise be considered darks.  It is likely due to violent deceleration when the payload encountered Earth's atmosphere.
		This effect can be seen as the small bump in the mean intensities at roughly 18:12:20 in Figure~\ref{fig:timeline}.  
		We restrict the dark frames to the data taken after the first two images, but before the shutter door opened in the upleg, and after the transient in the downleg; these times are shaded with gray in Figure~\ref{fig:timeline}.  
		In total there were \numDarkFrames\ usable dark images.  
		
		
		
		\begin{figure}[ht]
			\begin{center}
				\includegraphics{esis_pointing}
				\caption{The reference AIA 304\,\AA\ data taken at \aianearapogee \ was used for determining the absolute pointing. The octagon indicates the ESIS FOV.}
				\label{fig:fov}
			\end{center}
		\end{figure}
	

		\begin{table}
		\begin{center}
			\caption{ESIS Flight Data Summary}
			\label{tab:data_info}
			\begin{tabular}{ll|ll}\hline
				Launch Date & \dateMission & Image Size  & \imageShape~pix\\
				Data Acquisition Time & \timeDataStart--\timeDataStop~UTC & Avg. Noise: & \readoutNoise\tablenotemark{a}\\ 
			    Pointing   &  $\approx$ \esispointing & Avg. Gain &   \gain \\
				Field of View  & $\approx$ \esisfov octagonal  & Exposure Time & 10\,s \\
				Roll & $\approx$ \esisroll CW & No. Light Exp: &\numDataFrames\\
			    Spatial  Plate Scale  &  \plateScale & No. Dark Exp: &\numDarkFrames \\
				Spectral  Plate Scale  &  \dispersion & \\
					\hline
			\end{tabular}
		\tablenotetext{a}{Analog-Digital Unit}
		\end{center}
		\end{table}
		
		



	
\section{Data} 

We have established several levels of data processing that are described in detail below.
Level 0 represents the raw data that was obtained by the four different cameras during flight.
Level 1 data incorporates %prepares raw CCD data for scientific work through 
a quadrant dependent bias and gain correction, cropping of non-active pixels, and dark subtraction.
Normally, for solar observatories like AIA, higher level data would be further corrected for solar pointing and roll by interpolating the data onto a common grid, referenced to solar coordinates.\footnote{See the Guide to SDO Data Analysis, \url{http://helio.cfa.harvard.edu/trace/SSXG/ynsu/Ji/SDO_Data_Analysis_guide.pdf}.}  For ESIS this step would be complicated, and would not be the best treatment of the data.  
% First, the effective pointing of the instrument changed from image to image due to thermal expansion of optical components during the rocket flight.  
First, the spatial-spectral overlap on the detector complicates the notion of image alignment. Second, interpolation inherent in such geometrical corrections degrades spatial resolution unnecessarily.  
Finally, the ESIS optics slightly distort the solar image, mainly due to anamorphic magnification by the gratings, which occurs along the dispersion direction for each channel. 
In fact, the relationship between position and wavelength in each pixel depends on distortion parameters which we expect to vary subtly across the field of view, so that the distortion cannot be undone without simultaneously inverting the spatial-spectral cube across all wavelengths.
Because of this complication, we define two additional data levels.  

Level 2 data will include updated header information with accurate coefficients of a non-linear and wavelength dependent coordinate mapping from solar coordinates and wavelength to each detector for each image. It will also be corrected for atmospheric absorption and contain a header keyword tracking the correction as a function of time for use in instrument noise models.
In addition, Level 2 data will be despiked, and store a mask for future respiking.
Level 2 data is a necessary step toward inverted data products that cover the full ESIS wavelength range. The Level 2 distortion model must incorporate not only the optical design parameters, but the consequences of small element-by-element misalignments that inevitably occur when an optical system is built up.  This model is in active development, hence the Level 2 product will not be described further here.
% but will not be further described here because the results are optical model dependent and currently in development.

In lieu of a complete Level 2 product we seek a shortcut to interpretation of the ESIS data, which  we will refer to as Level 3.
Level 3 data is comprised of a single spectral line cropped from Level 1 data and mapped to the sky plane via a non-linear coordinate transform derived from a coalignment to co-temporal AIA data and an internal coalignment of each ESIS channel to a reference channel. Over the short wavelength range corresponding to a single spectral line (for our present purpose, we will choose \ov), it is quite straightforward to characterize the distortion by comparing the ESIS data to co-temporal AIA 304\,\AA\ data. This shortcut allows us to take differences between images and perform simplified inversions that are nevertheless rich in physical detail.

%This distortion can be measured and corrected manually, at least for the \ov \ data, by comparing the ESIS data to co-temporal AIA 304\,\AA\ data.  
   
% Instead it will be presented in a forthcoming paper.  
% Level 2 data will be released at the publication of that paper.  

    
The highest level data product we envision providing in the future, Level 4, is the spatial-spectral data cube, i.e., the intensity as a function of $[x, y , \lambda]$, which will be the results of various inversion methods.    
Since inversions generally do not produce unique solutions, there may be multiple Level 4 products obtained by different inversion methods.
% Level 4 data products will be described in future publications as they become available.   

The Level 0, 1, and 3 data products are described in detail in the remainder of this section. 

% Level 1 and Level 3 data will be made publicly available through various solar data archives. \cck{Such a vague promise is not much use. Are we ready to contact VSO?} 
    
\subsection{Level 0 Data}
    
Level 0 data is the raw data collected during the ESIS rocket flight.  
Each image was written to a fits files with on-board timestamp, camera number, and other parameters, such as the read out from temperature sensors included in the header.   

Each camera includes a detector with 2048 columns and 2064 rows.  The detector is operated in frame transfer mode, meaning the exposed portion of the detector is shifted into a storage portion that is read out during the next exposure. The storage portion is simply a mechanically masked region of the detector. The intention was that the storage region would be 520 rows per quadrant while the exposed region would be 512 rows per quadrant, producing an image that was 1024 rows in total.  An oversight during implementation inverted the size of the storage and exposed region, meaning the final image had 1040 rows.  Having a larger exposed region than storage region implies that eight rows originally exposed at the central portion of the detector were not behind the mechanical mask when the read out was initiated. However, because these rows were only unmasked for roughly 20\,ms, as compared to the 10\,s exposure time, there was no discernible impact on the data. 

% \cck{Deleted comment about low solar flux, which is irrelevant.}

The data is read out through 4 ports, corresponding to the four quadrants of the detector.  The read out register has 50 additional dummy pixels, or non-active pixels (NAPs), that are read out before each row of the detector data.  When the detector is read out, those 50 pixels are stored as part of the image.  Two additional ``overscan'' pixels are read after each row of data and also stored as part of the image.  A stored Level-0 data file, then, is 2152$\times$1040 pixels.   During pre-flight lab testing, it was found that the median intensity in columns 21-50 of the NAPs of each read out port are an excellent proxy for the detector bias.  Additionally the data from each port is converted from analog to digital through a unique circuit.  Though the same components were used, slight variations in the circuits caused slight differences in the gain (elec / DN) and read  noise associated with each quadrant.  The gain for each quadrant in each camera was measured prior to flight.  The noise was measured from usable dark images during flight.  The average and standard deviations in these values are given in Table~\ref{tab:data_info}.
    
\subsection{Level 1 Data}
	    \begin{figure}
	    	\centering
	    	\includegraphics{L0_to_L1}
	    	\caption{The top panel is the raw, Level 0 image captured by Channel \defaultChannel\ of ESIS during apogee. The bottom panel is the same image after Level 1 processing. The Sun viewed through the octagonal field stop in the \ov \ spectral line is on the right side of the detector, this portion of the detector was cropped to generate Level 3 data.  The Sun viewed through the octagonal field stop in the \hei \ line is partially on the left side of the detector. The \mgxbright \ line is between and partially overlapping both of these two strong lines.}
	    	\label{fig:L0_to_L1}
	    \end{figure}
    	
A summary of the flight data parameters, as described in the preceding sections, is provided in Table~\ref{tab:data_info}. 
For each channel, there were \numDataFrames\ light images that were processed into Level 1 data and \numDarkFrames\ dark images used to process the data.
Our procedure to convert from Level 0 to Level 1 is described below. An example of a Level 0 and Level 1 image is shown in Figure~\ref{fig:L0_to_L1}.

\begin{enumerate}
    \item Calculate and subtract the bias of each quadrant of each usable image (light or dark) by taking the median of columns 21-50 of the NAPs for each read out port.   
    \item Create a master dark image for each channel by taking the median along the time axis of all the bias-subtracted usable dark images.
    \item Subtract the master dark from each bias-subtracted light image.
    \item Crop each image to remove the non-active and overscan pixels.
    \item Convert images from DN to electrons by multiplying each quadrant of each image by the gain for that quadrant.
\end{enumerate}
The resulting Level 1 dataset contains each channel's image sequence in units of photoelectrons.
In addition to the images, the Level 1 fits header is updated with the time, exposure length, and altitude associated with each image using standard FITS header keywords.   
	

\subsection{Level 3 Data} \label{sec:level 3}
 
    
\newcommand{\vigfit}{[.35, 0.28, 0.34, 0.6]}
\newcommand{\levthreetime}{18:08:46}

The ESIS Level 3 data product is created to provide a co-aligned, single wavelength image in each channel for quick identification of events with significant line-of-sight velocities and easy comparison with coordinated data that does not require inversion. 
Level 3 data also allows for single wavelength inversion prior to the completion of the final optical distortion model and the Level 2 data product.  A demonstration of both qualitative and quantitative analysis of Level 3 data is given in Section 4.  

Level 3 data is generated directly from Level 1 as follows:
\begin{enumerate}
    \item Level 1 data is converted to units of photons in the target wavelength and then despiked.\label{step:photons}
    \item Optical distortion and internal co-alignment are corrected, and each channel is resampled onto an appropriately cropped common grid, with known alignment to AIA data.\label{step:distortion}
    \item Each channel is corrected for vignetting.\label{step:vignetting}
\end{enumerate}
Figure \ref{fig:coalign}a shows an example of a Level 3 image from Channel 2 taken at \levthreetime. All three steps make the assumption that the intensities are associated with the target wavelength. No attempt is made to distinguish photons of other wavelengths that may overlap with the octagonal field stop image in the target wavelength, so Level 3 is a useful construct for a target wavelength corresponding to a strong spectral line with minimal contamination. The above steps are described in greater detail below.
% Generating Level 3 data products requires several steps including 1) converting to  Level 1 data photons and despiking, 2) correcting the optical distortion and internal co-alignment of the four ESIS channels, 3) correcting of each channel for vignetting, including identifying regions where the contributions of the overlapping \mgxbright \ line confuse the \ov \ data, and 4) correcting the relative radiometric differences in each channel.  Each of these corrections are described in detail below. 



  \begin{figure}[htb!]
	\centering
	\includegraphics{aia_coalign.pdf}
	\caption{(a) An example of the Level 3 data product from Channel 1 taken at \levthreetime. This data has been aligned using nonlinear mapping parameters to the closest co-temporal AIA 304\,\AA\ image, shown in (b). }
	\label{fig:coalign}
\end{figure}
    	
    
\subsubsection{Conversion to Photons and Despiking}
In order to use Level 3 data for early inversion the intensity needs to  converted to photons in order properly account for uncertainty in the data.
The conversion from electron to photon is done to Level 1 data prior to co-alignment efforts.
The intensity in photons, $I_{\gamma}$, then becomes
\begin{equation}
   I_{\gamma} = I_{\SI{}{\elementarycharge}} * 3.6 \frac{\SI{}{\electronvolt}}{\SI{}{\elementarycharge}} * \frac{\lambda}{hc},
\end{equation}
where $I_e$ is the Level 1 intensity in electrons, and $\lambda$ is the target wavelength.
With a Silicon band gap energy of \SI[per-mode=symbol]{3.6}{\electronvolt\per\elementarycharge}, and an energy per \ov \ photon 
of 19.6\,\SI{}{\electronvolt} gives 0.18 photons per electron.
% Creating Level 3 data for a different spectral line in the ESIS passband simply requires using a different wavelength when converting to photons.

% \cck{I found the original despiking description confusing and difficult to read. Here's my attempt at a rewrite. Please check for technical accuracy (I relied on memory as well as text). Do you find it clear--if I only made it clear to me, I failed!} 
Due to geomagnetic activity on launch day, our detectors received a substantial number of charged particle hits. We therefore elected to despike the data.
% Due to a substantial number of charged particle hits on the ESIS detectors Level 1 data was also despiked in order to minimize ill effects on co-alignment and inversion.
Available despikers were difficult to tune in such a way that they eliminated particle hits, but not small transient events (e.g. explosive events). We therefore developed a new despiking algorithm, which is identified in Appendix \ref{despike}.






        
\subsubsection{Optical distortion correction and channel co-alignment}
   
   The four ESIS channels were spatially co-aligned in two steps.
First, each ESIS image is cropped around the \ov \ spectral line, (roughly pixels 1000 - 2048 of the Level 1 data shown in Figure~\ref{fig:L0_to_L1}) and then co-aligned to the closest AIA 304\,\AA\ image in time.  
The AIA 304 channel was chosen for co-alignment because it is the AIA EUV channel most visually similar to O V, in both the background and bright events (Figure \ref{fig:coalign}b).
Prior to co-alignment each AIA image was prepped to Level 1.5 using the aiapy routines \texttt{aiapy.calibrate.update\_pointing()} and \texttt{aiapy.calibrate.register()}.
The co-alignment was achieved through a linear coordinate transformation of the cropped ESIS image that maximizes the zero lag cross-correlation between it and AIA 304, the results of this are shown in Figure \ref{fig:coalign}.

Since ESIS has a slightly non-linear distortion function \citep{ESIS}, an additional internal alignment step is performed. 
Sub-pixel accuracy of the ESIS inter-channel alignment is critical to our tomographic inversion of the data to obtain line profiles, so we exercise more care than with the alignment to AIA.
Using a single ESIS channel as reference, in this case Channel 2, each other channel is co-aligned to it via a quadratic coordinate transformation that maximizes the zero-lag cross-correlation. 
Figure \ref{fig:cc} shows the ratio of peak cross-correlation after the quadratic internal alignment to that of the linear co-alignment with AIA for every camera pair and every exposure
A ratio greater than 1 implies this second step improved co-alignment.
We find that the internal alignment improves not only 
%After performing this additional internal alignment we find that not only is 
the co-alignment between each channel and the reference channel (dots in Figure \ref{fig:cc}), 
but also the co-alignment between every other combination of channels (stars in Figure \ref{fig:cc}).
This ratio is greater than 1 in all cases and shows a less than 1 percent improvement in peak correlation, demonstrating the subtle non-linearity of the ESIS optical distortion function.
In pixels, this corresponds to an average change in mapping of $\approx 0.4$ pixels.

After the co-alignment is complete each Level 3 image has been re-binned to AIA resolution, 0.6 arcsecond per pixel, and can be assigned the WCS information \citep{WCS} from the nearest image AIA in time, providing pointing information and easier co-alignment with coordinating instruments.



 \begin{figure}[htb!]
	\centering
	\includegraphics{internal_align.pdf}
	\caption{For each ESIS exposure (or image sequence) every channel pair, labeled in the legend, is cross-correlated to measure internal alignment quality.  The ratio of zero lag cross-correlation after a quadratic transformation to that of a linear transformation is plotted.  Every point is above the ratio = 1 line, indicating improved internal alignment for every combination of ESIS channels at each exposure.}
	\label{fig:cc}	
\end{figure}

 \begin{figure}[htb!]
	\centering
	\includegraphics{vig_correct.pdf}
	\caption{Panels a and c show example difference between images with \mgxbright \ masked that are uncorrected and corrected for vignetting respectively.  The line fit to the column mean as a function of column (fit between the red bars) shows a difference in background of $\approx 5$ photons before correction (panel b) at the edges of the field of view, and .4 photons after (panel d). }
	\label{fig:vig_correct}
\end{figure}

\subsubsection{Vignetting correction}
  
Each ESIS channel has a linear trend in intensity across the octagonal field of view along the dispersion direction due to internal vignetting in the optical system caused by having multiple stops in the optical system \citep{ESIS}.
In the co-aligned Level 3 images, where solar north is rotated to the top of the image, the dispersion axes run at different angles relative to solar north in each channel, so the impact of the vignetting field can be seen in the linearly trending background when taking the difference between two channels (Figure \ref{fig:vig_correct}a).
We corrected the vignetting by dividing out a linearly trending background, aligned to the dispersion axis, in every image. The parameters of this linear trend in each channel were obtained by minimizing the intensity trend seen in Level 3 difference images.

We define the vignetting field for the $i$th channel, at each time index, $s$, as a function of the pixels in the Level 3 data, $(x,y)$, as 
	\begin{equation}
		V_{is}(x,y) = \frac{m_i}{r_t} * [r(x,y,s) - r_0] + 1,
		\label{eqn:vignet}
	\end{equation}
where
	\begin{equation}
		r = x_0 + [\cos(\alpha_i)(x-x_0-x_{\text{drift}}*s/s_t) - \sin(\alpha_i)(y-y_0-y_{\text{drift}}*s/s_t)].
		\label{eqn:vignet2}
	\end{equation}

\cck{The first $x_0$ in the above equation appears to be an error.}\jdp{I just double checked this and I think it is needed.  Comes into play from translating the origin, rotating about (0,0), then translating back.  I believe it looks funny because the original field doesn't depend on y.  Let's walk through the math together to make sure.} In Equation \ref{eqn:vignet}, $r_0$ is equal to 65 pixels, and represents the distance from the Level 3 image edge to the ESIS field stop octagon edge at $s = 0$, the first Level 3 image.
The width of the octagon along the dispersion direction, $r_t$, is equal to 1140 pixels.
In Equation \ref{eqn:vignet2},  $\alpha_i$ is the angle of rotation of each ESIS Level 3 image relative to a Level 1 image row.
In this case, $\alpha_i = [112.5^{\circ}, 67.5^{\circ}, 22.5^{\circ}, -22.5^{\circ}]$ for Channels 1-4, respectively.
The vignetting field is rotated about the point $[x_0, y_0] = [635,635]$, which is the nominal center of the Level 3 image, to account for the change in dispersion direction.
ESIS images have a slight pointing drift, so the field of view defined by the octagon moves in Level 3 data as a function of time index, $s$. 
The empirical drifts are $[x_{\text{drift}},y_{\text{drift}}] = [8, -4]$ pixels. 
The total number of Level 3 exposures is $s_t=29$. \cck{Question: Why does the image drift come into the vignetting fit? Even if each exposure showed a completely independent, wouldn't the method work fine (in fact, probably even better)?}\jdp{This method assumes the vignetting function across the field stop octagon is constant in time for each channel.  Since the ocatgon drifts in solar x and solar y the vignetting field must drift with it.}
%Therefore the vignetting field is translated to follow this drift.
%This correction is achieved through a change in origin.
%At each time $s$, the origin is translated by $[x_{drift},y_{drift}]*s/s_t = [8_{pix},-4_{pix}]*s/29$, where $s_t $ is the total number, 29, of Level 3 images in time. \amy{why do you use st on the left side of the equation and 29 on the right?  Why not st both sides.  I guess I am questioning this equation, s/st is on both sides.  I think it only needs to be on the RHS?}\jdp{Charles, what do you think of this?  I am attempting to show that there is a linear pointing drift as a function of time in the x and y axis.  I'm worried if I plug numnbers in that point will be lost.  As it stands the left side here matches Equation \ref{eqn:vignet2}, and the right hand side shows the values for $x_{drift}$, $y_{drift}$, and $s_t$ filled in.}

%Due to inconsistencies between predicted vignetting field slope and what is seen in the data we choose to fit the linearly trending background in each difference image and use the vignetting field slopes, $m_i$, as free parameters to minimize it.
%We measure the trend in the background by taking the mean of each column in the difference image and fitting a line to it as a function of column position, shown in the right hand column of Figure \ref{fig:vig_correct}.

In the above relations, we have just four free parameters, the slopes $m_i$. 
The four channels of ESIS give 6 possible difference images for fitting the vignetting function for each image sequence $s$. 
When the average slope of all 174 combinations (6 difference images per each of the \numDataFrames \ exposures) is minimized we consider the vignetting corrected.

Measuring the trend from the data is additionally complicated by the overlap of the bright \mgxbright \ line.  
This line overlaps different regions of the field of view in each channel, see Figure~\ref{fig:mgx_overlap}.  
Therefore when fitting the background we restrict ourselves to the portion of the field of view without the \mgxbright \ overlap in any channel.
We also only use column means between the red bars in Figure \ref{fig:vig_correct} to ensure sufficient pixel numbers in each column.  

The resulting final fit, $m_i = \vigfit$, differs from the field predicted using ray tracing and geometric optical models \citep{ESIS}, which can be attributed to a few likely culprits. 
% One source of error likely comes from the imprint of adjacent spectral lines, the most obvious being that of \mgxdim \ visible in Figure \ref{fig:vig_correct}c.
% Since this Mg {\sc x} line overlaps almost entirely with O {\sc v}, and has an identical vignetting function, it adds intensity that prevents a perfect fit. 
% If this were the only source of discrepancy between the vignetting function predicted by the raytrace and the fits obtained from the data, then we would simply use the same, predicted vignetting for every channel. 
% However, we can anticipate slightly different vignetting in each channel due to variations in assembly so we allow each channels slope to vary independently.
% A misalignment of the ESIS field stop center, the ESIS primary optic center, and the center of the ESIS grating array, all shift the geometry of the ESIS central obscuration and can easily modify the vignetting field in each channel.
% Despite the uncertainties in vignetting, which we have found difficult to quantify, Level 3 differences are much flatter in intensity post vignetting correction as is seen in Figure \ref{fig:vig_correct}c, and will therefore lead to a higher fidelity intensity recovery when inverting Level 3 data.
% The resulting final fit, $m_i = \vigfit$, differs from the field predicted using ray tracing and geometric optical models \citep{ESIS}, which can be attributed to a few likely culprits. 
The vignetting function calculated by our optical models assumes a perfectly built and aligned ESIS instrument, resulting in each channel having identical vignetting.
A misalignment of the ESIS field stop center, the ESIS primary optic center, and the center of the ESIS grating array, all shift the geometry of the ESIS central obscuration and can easily modify the vignetting field in each channel.
% This misalignment would result in each channel having a vignetting field with a different slope.
When performing this vignetting correction we found we needed to allow the slope to vary for each channel to achieve a quality fit.
Another difference between the fit and the model is the slope of each vignetting field.
Our fits to Level 3 data show less vignetting in every channel than in predicted by our optical models. 
This likely comes from the imprint of adjacent spectral lines, the most obvious being that of \mgxdim \ visible in Figure \ref{fig:vig_correct}c.
Since this Mg {\sc x} line overlaps almost entirely with O {\sc v}, and has an identical vignetting function, it adds intensity that prevents a perfect fit. 
It is also possible that there is a subtlty to the ESIS vignetting calculation that we have failed to account for in our ray trace and geometric optical optics.
Despite the uncertainties in vignetting, which we have found difficult to quantify, Level 3 differences are much flatter in intensity post vignetting correction as is seen in Figure \ref{fig:vig_correct}c, and will therefore lead to a higher fidelity intensity recovery when inverting Level 3 data.

\cck{This presentation seems to say that the only reasonable cause for the major discrepancy in vignetting (between design and typical fits for $m_i$) is the various systematic errors in fitting. We are basically contradicting ourselves here, and leaving our readers in a quandary. I think the reasonable courses of acton are: (1) Stick to our decision (for L3 anyway), and don't mention the raytrace calculations which are evidently wrong for reasons that we do not understand. (2) Change our minds and apply a fixed vignetting slope based on the raytrace, under the assumption that the disagreement is a result of systematic errors from other spectral lines.}
\jdp{Based on our discussion Friday I reorganized the discussion of vignetting correction error.  Let me know what you think.}
        
  
        
        \begin{figure}[htb!]
        	\centering
        	\includegraphics{mgx_overlap.pdf}
        	\caption{The red hashed region shows the overlap of the \mgxbright \ spectral line in each channel that is masked prior to the vignetting correction and intensity normalization. These images are displayed in log scale in an attempt to bring out the subtle contamination.  The outline of the Mg\,{\sc x} can be seen very clearly in difference images like Figure \ref{fig:l3_dif}. }

        	\label{fig:mgx_overlap}
        \end{figure}
        
        

        
    \subsubsection{Relative Radiometric correction }
        The intensity of each channel is normalized by by equalizing the image mean over the least contaminated shared piece of Sun, and is therefore performed after inter-channel co-alignment and the vignetting correction.
        We divide each image by its mean in the region of the field of view uncontaminated by \mgxbright \ in every channel (shape shown in Figure \ref{fig:vig_correct} a and c) and multiplying it by the mean in that same region from the brightest image in the brightest channel (Channel 2 at apogee)
        This not only normalizes the intensity between each channel at every exposure, but also corrects for the effects of atmospheric absorption mentioned in earlier sections. 


\section{Preliminary Results}

        \cck{This paragraph does not belong here. Would go well in conclusions. Mention DeForest in intro (if not there already).}
	   ESIS is an extremely unique instrument.  It is not only a slitless imaging spectrograph, of which there only a handful of examples, it is also a Computed Tomography Imaging Spectrograph (CTIS).  
	   To our knowledge, the ESIS and MOSES sounding rocket instruments are the only CTIS instruments ever flown to observe the Sun, and  the Advanced Stokes Polarimeter on the Dunn Solar Telescope at the National Solar Observatory is the only ground based CTIS to capture solar data \citep{deforest2004}.  
	   MOSES only took three projections of the spatial-spectral data cube by observing $\pm$ 1 and 0 orders.
	   %, so its data is relatively simple when compared to the ESIS data set.   
	   ESIS added an additional projection, the ability to independently focus each channel, and a explicit field of view defined by the field stop, a significant upgrade in flexibility and data quality over MOSES.
	   
	   In this section, we provide a preliminary analysis of the ESIS Level 3 data. Our aims are to assess the ability of the ESIS data to (1) provide velocity signatures of dynamic events and (2) aid in developing a useful qualitative and quantitative understanding of the events.
	   Despite the lack of solar activity, ESIS managed to capture tens of small, transient events and one larger event during the $\approx 5$ minutes of observation in the \ov \ spectral line.  
	   We analyze five example events in this section.  
	   % Additional analysis is provided in a series of upcoming papers.  
	   
	   %\cck{The order
	   %should generally be to ask questions, describe/analyze data,
	   %and then draw conclusions. Original wording (below) seemed to
	   %beg the question.
	   %``...to both demonstrate that the ESIS mission accomplished its scientific goal of detecting the velocity signatures of small scale eruptive events, but also to provide useful qualitative and quantitative understanding of this data.''} 
		
	   
    \subsection{Level 3 Difference Images} \label{sec:dif_images} 
    	Early work with MOSES images demonstrated the utility of examining differences between different projections of the spatial-spectral cube (channels) to identify solar features with significant line-of-sight velocity \citep{Fox10,FoxPhD,RustPhD,Rust2019}, and to diagnose spectral contamination \citep{RustPhD, Rust2019}.
    	It is for this reason that we developed a Level 3 data product spatially co-aligned on O\,\textsc{v} %quickly 
    	that would allow us to take differences between ESIS channels.
    	Each ESIS channel disperses solar features in a different direction relative to solar north, determined by the orientation of each grating.
    	The positive dispersion direction in each Level 3 image is indicated by the arrows  in Figure \ref{fig:mgx_overlap}.
    	%In the case of an eruptive event with a velocity signature, the Doppler shifted intensity 
    	
    	Figure~\ref{fig:l3_dif} shows the difference between Channels 2 and 3 for the full ESIS field of view.  
    	Since wavelength is dispersed a different direction in each channel, taking the difference between two
    	simultaneous, spatially aligned channels cancels the intensity in the O\,\textsc{v} spectral line core. 
    	What remains are signatures of Doppler shifts, line broadenings, and other spectral lines.
    	Before we describe signatures of Doppler shift, it may be helpful to point out certain large-scale features that arise from spectral contamination.
    	The % white 
    	yellow octagon that overlays the lower left quadrant of the image is the \mgxbright \ line that overlaps the \ov \ line in Channel 2. Similarly, a dark octagon on the lower right quadrant is the portion of the \mgxbright \ line that overlaps the \ov \ line in Channel 3.
    	
    	
    	%leaves only intensity away from the O\,\textsc{v} spectral line core.  
    	%To put it another way, subtracting the data from two channels removes all the spatial structures that overlap in the two channels and leaves  easily identifiable velocity signatures. 
    	%For our initial analysis, we focus on the dominant O {\sc v} 629.7 \AA \ spectral line in the ESIS passband. \amy{I think this is a given since it is the only one in level3}
    	In the difference image,
    	Small-scale black and white lobes frequently appear near the locations of bright features on the Sun. A few of such features are marked with red boxes and will be discussed in detail below.  
    	The small black lobes tend to lie at an angle of $-22.5^\circ$ with respect to solar north, which is the direction of dispersion of Channel 3, while white lobes tend to lie at an angle of $+22.5^\circ$ with solar north, which is the direction of dispersion of Channel 2. A sufficiently compact and well-isolated brightening functions as its own spectrograph slit. 
    	The lobes then form V-, $\Lambda$-, or X-shaped features. % that indicate red shift (V), blue shift ($\Lambda$), or line broadening (X). 
    	Furthermore, intersection of the lobes occurs at the line center. Lobes above that intersection (i.e., along the dispersion direction in both channels) indicate a red shift (V). Lobess below line center similarly indicate a blue shift ($\Lambda$). The complete X-shape indicates a line broadening in which both red and blue shifts are present.
    	
    	Although the ESIS geometry differs significantly from MOSES, which had three images taken at spectral orders $m=-1, 0, 1$
    	of a single grating, analogous Doppler signatures appeared in MOSES data. These Doppler signatures were analyzed in great detail by \citet{Rust2019}, who sliced through small events along the dispersion direction to measure \heii \ line profiles from MOSES data.
    	% Every X or V-shaped features in an ESIS difference image, Figure~\ref{fig:l3_dif}, that have obvious, and nearby, positive and negative portions indicate solar events with significant line-of-sight (LOS) Doppler velocity.   
    	% Other things to 
    	    
   		
   		\begin{figure*}[htb!]
   			\centering
   			\includegraphics{l3_dif}
   			\caption{Full FOV Difference between Channel 2 and Channel 3 Level 3 images.  
   			Arrows denote the postive dispersion direction in Channel 2 and 3.
   			Events a-c are highlighted in Figure \ref{fig:dif_events}.    
   			Inverted results for Events c and d are shown in Section \ref{sec:inversions}. 
   			A time series of Event e is shown in Figure~\ref{fig:main_event}.}
   			\label{fig:l3_dif}
   		\end{figure*}
   	
   	
 		\begin{figure}[htb!]
   			\centering
   			\includegraphics{dif_events}
   			\caption{Events a, b and c identified in Figure \ref{fig:l3_dif} are examples of a mostly blue, red, and broadened even, respectively. The difference between Channels 2 and 3 is shown in the left hand column, while the Channel 2 and Channel 3 intensities are shown in the middle and right columns.  The velocity signature is most easily seen in the difference image, while the straight intensities provide information on the line core intensities. 
   			}
   			\label{fig:dif_events}. 
   		\end{figure}

    	%Simple, point-like, transient brightenings with little or no spatial structure are the easiest events to interpret because their naturally confined spatial extent acts similarly to a spectrographic slit.
    	%For that reason any stretching along the dispersion direction in an ESIS image can be interpreted directly as velocity. 
    	%In these simple events, a qualitative understanding of their velocity can be ``read off'' of each ESIS difference image.
    	%This effect was explored in great detail by \citet{Rust2019}, who sliced through small events along the dispersion direction to measure \heii \ line profiles from MOSES data.
    	
    	Three features, labeled a, b, and c in figure \ref{fig:l3_dif}, serve to illustrate the $\Lambda$, V, and X morphologies described above.
    	% Here we explore the velocity signature in three examples of simple point-line events marked a-c in Figure~\ref{fig:l3_dif}.  
    	Figure \ref{fig:dif_events} shows the difference image in the first column for all three events, then the Channels 2 and 3 data in the subsequent columns. 
    	Figure \ref{fig:dif_events}a shows a $\Lambda$-shaped event with lobes pointed downward in the difference between the Channel 2 and Channel 3 Level 3 image.
    	Since we know that the direction of positive wavelength dispersion is up and to the right in Channel 2 and up and to the left in Channel 3 (Figure \ref{fig:mgx_overlap}) we immediately know that this event is predominantly blue shifted.  
    	Similarly, an upward facing V-shaped event, like the one shown in Figure \ref{fig:dif_events}b, is predominantly red shifted.
    	X-shaped events, shown in Figure \ref{fig:dif_events}c, suggest enhanced emission in both the red and blue wing of the line profile. 
    	Difference images can provide immediate qualitative understanding of these simple, point-like events.  
    	During the 5 minute rocket flight, tens of similar events were detected in the ESIS field of view.  
    	We observe that the X-shaped events are especially common in this dataset, though multiple examples of each 
    	type are apparent.
    	
    	%While this gives us a quick and qualitative understanding of a simple event, even the smallest amount of spatial extent in a given feature leads to an entanglement of spatial and spectral information making it very difficult to derive qualitative velocities without inversion. 
    	
    	\begin{figure}[htb!]
    		\includegraphics{main_event_dif}
    		\centering
    		\caption{The largest and most complex velocity event %eruption
    		captured by ESIS shown at three different times (complete evolution in animation). The event starts as two jets (one red, one blue) with slight spatial separation, evolves into a strong red shift where the event begins paired with faint blue shifted material above, and ends in a complicated combination of velocity not easily interpreted from difference images. }
    		\label{fig:main_event}
    	\end{figure}
		
    	During the ESIS flight, we also captured a handful of spatially extended Doppler signatures that are more difficult to interpret.
    	The most obvious of these is an eruption near disk center, Event e in Figure \ref{fig:l3_dif}.
    	This large event is shown at three times in Figure \ref{fig:main_event}.
    	The enganglement of spatial and spectral information in this relatively complex event makes it difficult to derive a qualitative interpretation from the difference images (left column of the figure).
    	%As shown, an erupting two-dimensional feature leads to an entanglement of spatial and spectral information that makes it very difficult to derive qualitative velocities without inversion.
    	% In difference images (left column) 
    	% there is a mess of positive and negative features intertwined 
    	A cluster of overlapping positive and negative lobes in the brightest portion of the event sometimes present as a $\Lambda$, V, or X shape, but not always, and evolve significantly in time.
    	The top row shows this event early in its evolution. 
    	Intensity is concentrated to a small region and presents as two jets, one red and one blue, with the red shifted source slightly to the right of the blue shifted source. % a slight spatial separation.
    	% Shortly after 
    	100 seconds later (middle row), the event is dominated by a significant red shift (upward facing V) at the location of initial brightening, 
    	There is also an imprint of fainter differences above the brightest knot of intensity showing motion along a spatially extended structure.
    	The difference lobes in this faint structure are opposite in polarity to the intensity below, and therefore 
    	appear to represent  blue shifted material ejected from the site of initial brightening.
    	By the end of the event (another 70 seconds later, bottom row) the dominant red shift subsides, leaving a complicated spectral signature at the brightest portion of the event, and in the adjacent region up and to the left.  
    	
    	So far, we have been exploiting two of the four ESIS channels with our difference image. This approach is qualitative,
    	and its limitations become apparent with complex and spatially extended dynamics as we saw in event e.
    	% Dynamic and spatially extended events are excellent opportunities for ESIS to shine and clearly demonstrate the need for many different projection angles (ESIS channels) if we hope to disentangle spatial information from spectra.
    	% Despite the extra complexity, 
    	An understanding of ESIS difference images provides a starting point,
    	and offers a sanity check when interpreting numerical inversions that take advantage of all four projections.
    	In the following section we will focus on inverting two smaller events (c and d), and will save a more in depth analysis of Event e for a future publication. 
    	
    \amy{I think you did a good job of decribing this event in the text.  You may want to add some arrows in the figure that you can refer to in the text for added clarity.} \cck{Good idea.}
    % 	Larger positive or negative features with no obvious counterpart nearby, are indicative of spectral contamination \citep{RustPhD,Parker2021}.
    % 	The most easily seen impact of this is a faint octagon edge visible in Figure \ref{fig:l3_dif} from Mg {\sc x} 625.9 \AA.
    % 	Extra spectral content will act as a source of error when inverting Level 3 data, but will be properly accounted for by a wavelength dependent optical distortion model and the completion of the ESIS Level 2 data set \citep{Smart2022}. 	 
    % 	\amy{roy's paper already slated for 2022?  I mention some of this above, don't know if we need to keep it in both places.}
    
    
    \subsection{Early Inversions} \label{sec:inversions}
    	In order to better disentangle the spectral and spatial information captured by ESIS, and to provide quantitative velocity information, all four channels must be combined and ``inverted'' to return a single, spatial-spectral cube, $I(x,y,\lambda)$, at every exposure.
    	For out preliminary inversions of the \ov \  ESIS Level 3 data we used a Multiplicative Algebraic Reconstruction Technique (MART).
    	MART is attractive for our first inversions because it is fast, requires no training or explicit assumptions about the data, and automatically enforces image  positivity.
    	We describe our MART implementation in % more detail in 
    	Appendix \ref{MART}.
    	
    	\begin{figure}[htb!]
   			\begin{tikzpicture}	
	    		%begin by adding a node for each figure
	    		\node[inner sep=0pt] (imgs) at (0,0)
	    		{\includegraphics[]{perfectx_inverta}};
	    		\node[inner sep=0pt] (lps) at (0,-10.4)
	    		{\includegraphics[]{perfectx_invertb}};
    		\end{tikzpicture}
%    		\includegraphics{perfect_x_inverted}
    		\centering
    		\caption{MART inverted results for event c in Figure \ref{fig:l3_dif}. The integrated intensity (top row) and corresponding Level 3 difference image (second row) are shown at 4 different times. A red box on each difference image marks the FOV used in the top row.  The \ov \ line profile at each position marked with a red dot is plotted in the matching 3x3 grid in a different color for each time (in order red, green, blue, black). Each MART line profile (dots) is overplotted with a double gaussian fit (solid line).  Bulk shifts in km/s for each component of the fit are provided in the legend for each time in their respective color. }
    		\label{fig:perfect_x_inverted}
    	\end{figure}
        
        \begin{figure}[htb!]
       		\begin{tikzpicture}
	        	%begin by adding a node for each figure
	        	\node[inner sep=0pt] (imgs) at (0,0)
	        	{\includegraphics[]{otherx_inverta}};
	        	\node[inner sep=0pt] (lps) at (0,-10.8)
	        	{\includegraphics[]{otherx_invertb}};
        	\end{tikzpicture}
%    	\includegraphics{other_x_inverted}
    	\centering
    	\caption{MART inverted results for event d in Figure \ref{fig:l3_dif} presented in the same fashion as Figure \ref{fig:perfect_x_inverted}.}
    	\label{fig:other_x_inverted}
    	\end{figure}
    	
    	In Figures \ref{fig:perfect_x_inverted} and \ref{fig:other_x_inverted}, and their associated animations, we present the MART inversion for two different compact transient events, labeled c and d in Figure \ref{fig:l3_dif}.
    	Both events have an X-shaped presentation, begin and end within the ESIS observing time, and show noticeable temporal evolution in velocity.
		Each figure shows the total \ov \ integrated intensity, %output by the MART inversion summed 
		obtained by summing the MART inversion
		along the wavelength dimension, 
		in the top row  at four different times. 
		The second row shows uninverted Channel 2-3 Level 3 difference images at those same times for comparison.
		The FOV of the intensity image from the top row is shown as a red box on the difference image. % below. 
		% Inverted \ov \ line profiles at each spatial location marked by a red dot in the integrated intensity image 
		% are plotted in the matching 3x3 grid for each time (in the order red, green, blue, black).
		Inverted \ov line profiles are plotted in a $3\times 3$ grid, corresponding to the grid of 9 positions marked
		in red on the intensity images. The red, green, blue, and black curves correspond to the four times indicated in
		the top row.
		We performed a double gaussian fit for each line profile in order to more easily pick out the location of each peak in velocity.
		The results of the fit (solid line) is plotted over the MART inverted data (dots).
		The deviation from line center of each gaussian is recorded in km s$^{-1}$ in the color matching the line profile.
		In the animation, we plot the red and blue components of each fit separately as dashed lines, their sum as a solid line in fuschia, and the MART inverted line profile in  green dots.
		
		Event c, shown in Figure \ref{fig:perfect_x_inverted}, lasts just over 4 minutes and shows significant temporal evolution in the line profile.
		Throughout the event the blue component of the \ov \ line dominates and is centered at $\approx -50$\,km\,s$^{-1}$.
		This blue component persists for the entire duration of the event with % very slight deviations 
		relatively gradual changes in intensity and velocity.
		The red component of the event is bursty in nature, with two small peaks in intensity at 18:07:36 and 18:08:16, and is centered at $\approx$ 120 km s$^{-1}$. 
% 		While there are a several frames where the red component of the line appears as an additional peak in the line profile and is well fit by a second gaussian, most of the time it presents more as an asymmetric broadening in the red wing of the line making the velocities from the fit less reliable. 
% CCK edit below, original text above.
	    The red and blue components of the line are only occasionally well separated, but the two Gaussian fit consistently follows the inverted line profiles with only small deviations.
		This explosive event is $\approx$ 3 arcseconds in diameter and shows little spatial variation in the line profile across the event.
		
		The explosive event highlighted in Figure \ref{fig:other_x_inverted}, event d, is similar to event c in that is has enhancements in both the red and blue wing of the line profile, but is more complicated in its presentation.
		Event d  begins around 18:07:16 and its intensity has almost completely died away by the last Level 3 image.
		At the begining of the event a slightly redshited component dominates the line profile.
		A strong blue component in the line profile centered near -130\,km\,s$^{-1}$ begins to appear 18:07:16 and peaks at 18:08:06. 
		This blue component is brightest in the lower left portion of the event.
		The red component of the line profile peaks at 18:08:56 and is centered near 50 km s$^{-1}$.
		Unlike event c, the blue component of the line is less persistent in time.
		Also, the blue and red components of the event do not occur in the same spatial location.
		This is visible both in the grid of line profiles and the difference images.
		Intensity in the line profiles show the blue component peaking in the lower left, while the red component peaks in the middle to right columns.
		In the difference images, this corresponds to a clear vertical and horizontal separation between the blue $\Lambda$ and the red V. 
		A closer look at the inverted data shows an $\approx$ 1.35 arcsecond spatial separation between centroids of the blue and red component.

		%Despite two free parameters in our MART algorithm, namely the exponent used on the contrast enhancement filter and the number of times it is applied (Appendix \ref{MART} steps \ref{step:contrast} and \ref{step:smooth}), we consistently find line profiles consistent with a qualitative interpretation of the difference image movies, and measure bulk flows within $\approx\pm$ 15\,km\,s$^{-1}$ of each other for a given exposure.
		%Figure \ref{fig:perfect_x_invertcomp} shows the results of tweaking these free parameters for a single timestamp of Event C and is paired with an explanation of their impact in Appendix \ref{MART}.
	    		   	
        There are two free parameters in our MART algorithm, namely the exponent used in the contrast enhancement filter and the number of times it is applied (Appendix \ref{MART} steps \ref{step:contrast} and \ref{step:smooth}). The inversions in this section were tried with a wide range of these parameters. The Appendix shows the results of tweaking these free parameters for a single timestamp of Event C.
        We find that the results are robust to variation in both parameters. In particular, we consistently find line profiles consistent with a qualitative interpretation of the difference image movies, and measure bulk flows within $\approx\pm$ 15\,km\,s$^{-1}$ (\cck{How many ESIS pixels is that?}) for a given exposure. 
        \jdp{approx 1 pixel in Level 3 data.  This number is really generous and we could probably tighten it up. Most of that error comes from the red curve where the results have tons of plaid and the fits aren't great.  I was just trying to not sugar coat anything.}
    	
\section{Discussion/Conclusions and Future Work}
	The ESIS sounding rocket mission, launched September 30th 2019, was successful in capturing spatial and spectral information over is entire \esisfov \ field of view in multiple wavelengths (\hei, \mgxbright, and \ov) at every exposure.
	We have processed the data into multiple levels such that preliminary scientific work can be done, included a Level 3 data product that allows for taking spatial differences between channels, and early inversion (Section \ref{sec:level 3}).
	Level 3 difference images in the \ov \ wavelength (Section \ref{sec:dif_images}) reveal a host of small transient brightenings across the field of view with significant line-of-sight Doppler velocity, some in excess of $\pm 100\,$km\,s$^{-1}$.
	They also reveal several larger and more complex events with large velocities, most notably the eruption identified near disk center (event e in Figure \ref{fig:l3_dif}).
	
    During the roughly 5 minutes of ESIS observing time it captured tens of events, evolving on 10 seconds timescales, over its 11.3\,arcminute field of view.
	Even seemingly simple explosive events evolve fast enough in time, and have significant spatial distributions of intensity and velocity, such that rastering a traditional slit spectrograph would fail to capture the event completely. 
	This combined with the fact that ESIS can measure tens of events like this simultaniously makes it a powerful tool for capturing and diagnosing small eruptions in the solar atmosphere.
	
	We use a Multiplicative Algebraic Reconstruction Technique (MART, Appendix \ref{MART}) to disentangle the combined spatial and spectral information in the four ESIS Level 3 images to find a single spatial-spectral cube $(x,y,\lambda)$, covering just the  \spectralline{O}{v}{630} line, for two small transient brightenings (Section \ref{sec:inversions}).
	Our inversions reveal strong red and blue jets with bulk flows in excess of 100\,km s$^{-1}$ that evolve on the timescale of the ESIS cadence, 10 seconds.
	These two events are spatially compact, a few arcseconds, and have lifetimes of a few minutes.
	Despite their compact nature, the red and blue components measured are separate and distinct in both space and time. 
	This is especially noticeable in event d (Figure \ref{fig:l3_dif} and \ref{fig:other_x_inverted}) where the centroids of the blue and red jet are displaced by a little over an arcsecond. 
	
	The bimodal nature of these small brightenings is similar to observations made by \citet{Rust2019} \heii \ using MOSES data.
	This presentation is quite different from line profiles observed by IRIS \citep{depontieu2014} in \spectralline{Si}{iv}{1394}, where small transient events show smaller bulk flows, dominant line core emission, and excessive non-thermal broadening.\jdp{gathering a few citations}
	Assuming such high bulk velocities are indicative of reconnection outflows \cck{(Rather than assuming, make the argument. Cite the Innes article in Nature, and other appropriate sources)}. 
	The bimodal profile suggests a very small reconnection region with little to no stationary emitting plasma, pointing away from a dominate tearing mode instability, and toward a Petschek type reconnection \cck{(citation needed--Innes \& Guo modeling work, at least)}.
	Though we hypothesize that both jets originate at a single reconnection site, we admit that it is puzzling that observed fluctuations of the red and blue jets are not temporally correlated, most notably in explosive event d (Figure \ref{fig:other_x_inverted}).  \cck{Tom found distinct peaks in individual MOSES projections. Did you uncover something similar with ESIS, or are we relying on the inversion for this? If the latter, mention the attempt to explore parameter space, looking for alternate interpretations.}
	It may be that the reconnection in these small events is more complex than we imagine; however, it is also possible that, looking in a single spectral line, we are not seeing all of the ejected plasma. 
	Further work with the ESIS data may help to clarify this issue.
	Several of these events were captured in \hei, \mgxbright, and \ov \ simultaneously, putting ESIS in a position to measure the outflows at multiple temperatures. 
	If the event is present in hotter or cooler lines we can also see if the bi-modal nature of these line profiles is unique to \ov \ or persists at lower and higher temperatures.
	Also, IRIS was running small four step coarse rasters near disk center during the hour long ESIS launch window.
	Although none of the events analyzed in this paper were captured by IRIS, a closer look at the IRIS data may reveal small explosive events suitable for direct comparison with ESIS line profiles.
	
	
	
	The next major tasks in the analysis of ESIS data will be to invert the entire spatial-spectral cube, from \spectralline{He}{ii}{584} to \spectralline{O}{v}{630}. 
	A self-consistent inversion will be required to separate out the dimmer \mgxdim, and perhaps even the faint \oiii \ line. This will require a careful characterization of the wavelength-dependent distortion in the images, which is currently underway.
	Future multi-wavelength inversions of the ESIS data will allow for more accurate estimates of the frequency and distribution of explosive events across a large portion of the solar disk and at different heights in the solar atmosphere than have been possible with slit spectrographs.
	It will also allow us to track energy and material moving through multiple layers of the solar atmosphere to form a clearer picture of where and how these reconnection events unfold. 
	
	\jdp{Do we need one more paragraph where we compare Event E to a mini filament eruption and cite a few papers?  Also, should we mention coordinated data from the mission (i.e. SOT/ESIS/IRIS ?)}\cck{I think yes on the first but probably no on the second. It's not too helpful to describe coordinating data and then not use it. However I do agree that's an important aspect of our next paper on this dataset.}
	

\begin{acknowledgements}
	This work was supported primarily by NASA Grant NNX14AK71G.
	Roy Smart acknowledges NASA NESSF program grant 80NSSC17K0524.
	We would also like to acknowledge the NASA SRPO and NSROC employees involved, all of whom went above and beyond to support ESIS from start to finish, we couldn't have done it without you. 
	CHIANTI is a collaborative project involving George Mason University, the University of Michigan (USA), University of Cambridge (UK) and NASA Goddard Space Flight Center (USA).
	This research used version 0.5.0 (10.5281/zenodo.4676478) of the aiapy open source software package \citep{aiapy}.
	
	
\end{acknowledgements}



\appendix
\section{Despiking}\label{despike}
    We developed a new image despiking algorithm with a spike intensity threshold that varies according to the local median intensity. The key to our strategy is to establish a finite number of bins for the local median intensity, and then examine the statistics of all the pixels falling into each bin. Our despiking routine is implemented in the following steps:
\begin{enumerate}
    \item For each data dimension (image row, image column, and time), calculate a local median intensity image by averaging the nearest 11 pixels along the given axis. Divide the range of median image values into 128 bins.
    \item For each local median bin, form a cumulative distribution of intensities. Identify the intensity value corresponding the 99.9th percentile. The result is three curves (one for each data dimension) of 128 points each. 
    \item For each data dimension, fit a line to the threshold curve. The line identifies the relationship between median intensity and spike threshold for that dimension.
    \item Create three masks, which are  boolean-valued images identifying pixel values that exceed the thresholds, one image per data dimension.
    \item Combine the three masks with logical AND to produce the final map of bad pixels (spikes). 
    \item Replace each bad pixel with a local mean of good (i.e., not bad) pixels only, weighted by kernel $k$ (Equation \ref{eq:despike_kernel}).
\end{enumerate}
The kernel used to determine the replacement value for bad pixels is
\begin{equation} \label{eq:despike_kernel}
    k(x,y,t) = e^{-\lbrace|x|+|y|+|t|\rbrace/a}.
\end{equation}
For our implementation $a=.5$\,pixels and $x,y,t= -2, -1, 0, 1 , 2$.
The kernel is normalized prior to application by it's total, ignoring any other bad pixels that may fall within the kernel.

The algorithm described above has a number of advantages compared to those available in standard data prep routines for solar physics missions. 
We are aware of only one published example of a despiker that uses the time axis for spike identification, used by \citet{Aschwanden2000(trace_unspike_time)} on data from the Transition Region and Coronal Explorer \citep[TRACE]{handy1999}. 
In our algorithm, a pixel must stand out from its neighbors in all three dimensions to be identified as a spike.
This criterion reduces false positives; yet because of this approach, we can choose a relatively sensitive percentile threshold based on image statistics to reduce false negatives. 
Usually, a percentile-based threshold would result in identifying a fixed number of bad pixels; but since the logical AND is nonlinear with respect to the three binary masks, it is possible to identify a small or large number of spikes in a given image. 
For example, with our 99.9th percentile threshold, the algorithm can find as few as zero or as many as 2129 bad pixels in a $2048\times 1040$ Level 1 image. 

\section{Multiplicative Algebraic Reconstruction Technique}\label{MART}
	Multiplicative Algebraic Reconstruction Technique (MART) was developed for limited-angle tomography problems, in which a small number of projections are used to reconstruct a volume \citep{Okamoto1991,Verhoeven1993}. 
	As ESIS has only four projections through the $(x,y,\lambda)$ volume, it is a good candidate for this approach.
	When testing a variety of inversion methods on MOSES data \citet{FoxPhD} identified MART as most promising of several methods in terms of speed and fidelity.
	\citet{RustPhD} applied a slightly different version of MART to the MOSES data paired with a wavelet based partial reconstruction technique for better background subtraction and event isolation.
	Due to its speed, lack of required training, and automatic enforcement of image positivity we have chosen MART for our first inversion of Level 3 ESIS data.
	
% 	\cck{Needs a mathematical problem description, which would include definition of the notation used below. I think all we need is an (idealized, without distortion, etc) equation for the projection, and a few words describing what is known or unknown. So, just the first equation of step 4, but with an $I$ instead of a $G$. Speaking of that equation, it seems to only make sense if you unprime the $x$ and $y$. Perhaps you meant to write $G_{\theta (x'-\delta\lambda) y'}$, in both equations in that step? You need to define $\delta$. When using L3 data, is $\delta=1$? If not, it seems that putting the coordinates as indices rather than floating point arguments might not make sense.}
	
	
	With MART we seek to reconstruct the true spatial-spectral cube, $I_{xy\lambda}$, using every ESIS image.
	Each ESIS image, $i_{\theta x'y'}$, can be expressed as an angled projection through $I_{xy\lambda}$ onto a 2-D detector, or,
	\begin{equation}
	    i_{\theta x'y'} = \sum_\lambda I_{\theta(x'-\delta\lambda)y'\lambda},
	\end{equation}
	where
	\begin{equation}
		I_{\theta x'y'\lambda} = \mathcal{R}_\lambda(\theta)\,I_{xy\lambda},
	\end{equation} 
	and, $\mathcal{R}_\lambda (\theta)$, is a rotation about the $\lambda$ axis into the primed detector coordinates $[x',y']$.
	There is one ESIS image for each $\theta$ representing the relative orientations of each ESIS channel.
	Projections through $I$ are done along lines of constant $x'-\delta\lambda$.
	For our specific implimentation, the resolution of $I_{xy\lambda}$ in $\lambda$ is chosen to be 28\,m\AA\,pix$^{-1}$, the spectral dispersion of an ESIS grating, such that $\delta = 1$. 

	MART attempts to find the true spatial spectral cube $I_{xy\lambda}$ by itteratively correcting a guess cube $G_{xy\lambda}$ until it's projected images, $g_{\theta x'y'}$, match each esis image, $i_{\theta x'y'}$, with a reduced chi-squared, $\chi_{R,\theta}^2$,  less than one.
	Multiple different $G_{xy\lambda}$ can satisfy $\chi_{R,\theta}^2<1$ so we impliment an additional outer filtering loop that enhances the contrast of $G_{xy\lambda}$ and applys smoothing prior to beginning additional MART corrections to further constrain our inversions.
	Our procedure is executed as follows:
	\begin{enumerate}
		\item \label{step:guess} Create a guess cube,
		initialized as 
		$G_{xy\lambda} = 1$ on the same domain as $I$. 
% 		\cck{(It looks like the total intensity is arbitrary in the initial guess, and has to eventually match observation through the correction factors. I don't think that's necessarily a problem, but it does modify the effect of the contrast parameter $\Psi$ as the total intensity converges.)}
		\item \label{step:contrast} Enhance the contrast of $G$ and normalize. 
			\begin{equation}
				G \leftarrow \frac{G+G^{(1+\Psi)}}{\sum_{xy\lambda}G+G^{(1+\Psi)}}\sum_{xy\lambda}G, 
			\end{equation}
		
		\item \label{step:smooth} Convolve $G$ with smoothing kernel $K$, $G \leftarrow G * K$,
		in our case,
			\begin{equation}
			\label{eq:kernel}
				K_{ijk} = \frac{2^{3-|i|-|j|-|k|}}{64} \quad \text{for}\quad i,j,k = -1,0,1.
			\end{equation}
		\cck{Don't think I realized you were doing this. It's
		very powerful smoothing, much stronger than what I can
		recall using in the past (ISTR using something like $\mathcal{I}+\epsilon K$, with $0 \le \epsilon \ll 1$). 
		But if $\chi^2_R$ can approach unity,
		I don't see any problem with it.}
	    \jdp{Better check me on this.  Looking at the code I see a 1-D kernel ,[.25,.5,.25], which is rotated and multiplied together to be 3-D.  I believe I represented it correctly in 3-D.  In your code Charles (moses.py) from the MOSES inversion the kernel is different.  In 2-D it is, kernel = np.array([[0,1,0],[1,4,1],[0,1,0]]).  In both cases they are simply convolved with the spatial-spectral cube}
		\item \label{step:project} Sum $G_{\theta xy\lambda}$ along lines of constant $x-\delta\lambda$ to calculate a projection for each angle $\theta$,
			\begin{equation}
				g_{\theta x'y'} = \sum_\lambda G_{\theta(x'-\delta\lambda)y'\lambda}, 
			\end{equation}
		\jdp{Should I set $\delta=1$ here since it is described above?}
		where
			\begin{equation}
				G_{\theta x'y'\lambda} = \mathcal{R}_\lambda(\theta)\,G_{xy\lambda},
			\end{equation} 
		and, $\mathcal{R}_\lambda (\theta)$, is a rotation about the $\lambda$ axis. 	
		
		\item \label{step:chisquared} Calculate reduced chi-squared for each projection and check for convergence, in this case that $\chi_{R,\theta}^2 < 1$ , 
			\begin{equation}
				\chi_{R,\theta}^2 = \frac{1}{N_{x'} N_{y'}}\sum_{x'y'} \frac{(i_{\theta x'y'}-g_{\theta x'y'})^2}{g_{\theta x'y'}+\sigma^2_{RN}},
			\end{equation}
		where $\sigma_{RN}$ is the read noise in photons, $N_{x'}$ is the total number of elements along $x'$.
		
		\item Calculate correction factors for each unconverged channel, $\theta_{uc}$, weighted by $\gamma$, 
			\begin{equation} \label{eq:correctionfactor}
				c_{\theta x'y'} = \left[\frac{g_{\theta x'y'}}{i_{\theta x'y'}}\right]^\gamma,
			\end{equation}
		where $\gamma = 2/n$ with $n$ equal to the total number of channels.
		
		\item Assign correction factors to lines of constant $x-\delta\lambda$ in the domain of $G$,
			\begin{equation}
				C_{\theta (x'-\delta\lambda)y'\lambda} = c_{\theta x'y'}
			\end{equation}	
		
		\item \label{step:correct} Apply a weighted product of each derotated correction to $G$ ,
			\begin{equation}\label{eq:correct}
				G \leftarrow G\left\lbrace  \,\prod_{\theta=\theta_{uc}}  \mathcal{R}_\lambda(-\theta_i)C_{\theta x'y'\lambda} \right\rbrace^{1/m_{uc}},
			\end{equation}
		where $m_{uc}$ is the total number of unconverged channels and 
		
		\item Repeat steps \ref{MART}\ref{step:project}-\ref{MART}\ref{step:correct}
		until every channel has converged at step \ref{MART}\ref{step:chisquared}. Once converged proceed to next step.
		\item Return to Step \ref{MART}\ref{step:contrast} if additional filtering is desired.
	\end{enumerate}
	When applied to  ESIS Level 3 data the total number of channels, $n = 4$.
	All negative values introduced by rotation, the result of ringing due to the Gibbs phenomenon associated with interpolation when rotating the data cube, are set to zero.
	In our case, this ringing occurs mostly at the sharp discontinuity at the edge of the field stop octagon. 
	
	\begin{figure}[htb!]
		\begin{tikzpicture}
		
		%begin by adding a node for each figure
		\node[inner sep=0pt] (imgs) at (0,0)
		{\includegraphics[]{perfectx_invert_comp_a}};
		\node[inner sep=0pt] (lps) at (0,-9.5)
		{\includegraphics[]{perfectx_invert_comp_b}};
		
		
		\end{tikzpicture}
		\centering
		\caption{Comparison of MART ouputs for a single timestamp in Event c (Figure \ref{fig:l3_dif} and \ref{fig:perfect_x_inverted}) with different amounts of filtering. 
		The top row shows the total inverted intensity at a single timestamp each using a different number of MART filtering steps (steps \ref{MART}\ref{step:contrast} and \ref{MART}\ref{step:smooth}).  
		The lower grid shows the inverted O\,\textsc{v} spectral line at each red dot for 0, 25, and 75 filtering steps in red, green, and blue respectively.}
		\label{fig:perfect_x_invertcomp}
	\end{figure}
	
	
    Our implimentation of MART has two free parameters, namely the contrast enhancement exponent $\Psi$ and the number of times the filtering step (steps \ref{MART}\ref{step:contrast} and \ref{MART}\ref{step:smooth}) is applied.
    The contrast enhancement and smoothing steps are designed to fight the tendency of MART to smear intensity along the dispersion directions in a way that matches the data, but results in artificial intensity in the far wings of the inverted spectra.
	Since the enhanced wing artifacts from bright features intersect and form a gridlike pattern of enhancements, we use the nickname ``plaid'' to describe
	this artifact.
% 	Each time the filter is applied (steps \ref{MART}\ref{step:contrast} and \ref{MART}\ref{step:smooth}) steps \ref{MART}\ref{step:project}-\ref{MART}\ref{step:correct} are repeated until convergence in every channel.
    In order to better understand their impact on our inversions we inverted a single exposure of Event c (Figure \ref{fig:l3_dif} and \ref{fig:perfect_x_inverted}) using several different values of each free parameter.
    During this exercise we found that lowering $\Psi$ resulted in faster MART convergence, but required more applications of the filtering step to minimize the plaid.
    Increasing $\Psi$ lowered the number of required filtering loops, but increased the convergence time of each MART iteration and quickly lead to overfiltered and artificial looking results.
    In the end we chose to use $\Psi=0.2$ in an effort to not overfilter our inversions and maintain a reasonable total algorithm convergence time.   
    
    Figure \ref{fig:perfect_x_invertcomp} shows the results of changing the total number of filtering steps with $\Psi=0.2$.
    While other numbers of filtering steps were tried, we chose to display the inverted results after 0, 25, and 75 steps since they sufficiently illustrate the effect of varying this parameter. 
	The impact of plaid is most evident in the red spectral lines in Figure \ref{fig:perfect_x_invertcomp} where the contrast enhancement filter wasn't used at all.
	Despite excess intensity in the wings of the lines from excessive plaid, the brightest pixels are still double peaked in nature and the two Gaussian fits do a decent job of fitting the inverted line profiles.
	The blue spectral lines in Figure \ref{fig:perfect_x_invertcomp} show line profiles after repeating the filtering step 75 times, which is enough times for the solution to reach an equilibrium that returns the same results once converged.
	Since the contrast enhancement filter is designed to pull intensity from the background and into regions of stronger signal, it has the effect of clumping the intensity into a smaller area.
	This causes much less intensity in the wings of the line profile and narrower peaks overall.  
	Unfortunately this also has the ill effect of clumping background intensity where there is little signal instead of allowing for a flat noisy background as expected.  
	Therefore we choose to compromise and run the filtering step enough times, in our case 25, such that the majority of the artificial intensity in the wings has been beaten down, but not so much that we artificially remove a flat noisy background in low signal regions.
	
	Regardless of which combination of parameters we chose, we find that our inverted results match qualitative interpretations of ESIS difference images.
	Applying more or less filtering does affect how bi-modal each spectral line presents and slightly changes measured width and bulk flows.
	Despite these difference we find surprising agreement in measured bulk flows from two Gaussian fits, even when including including spectral lines with zero filtering and a maximum amount of plaid (red lines in Figure \ref{fig:perfect_x_invertcomp}).
	
	The final configuration we used when inverting all Level 3 data used 25 filtering steps and a constrast enhancement exponent of $\Psi=.2$.
	In this configuration, a single exposure of a small region like Event C (with final inverted dimentions of $[x,y,\lambda] = [140,140,41]$\,pixels) can be inverted in $\approx$75\,seconds on a single thread of a four core 2.8\,GHz processor.
	Since the inversion of each ESIS exposure is independent, inversions can be done on multiple threads to increase efficiency when inverting additional exposures.
	
	\cck{Need to introduce the parameter variation exercise and its motivation, and simply describe the figure, indicating which parameters were varied and how for all three examples, before diving into its interpretation. Do the three examples given here suffice to show that we have explored the 2 free parameters in the inversion? Would it be useful (and accurate) to indicate that many more combinations have been tried, but these examples are shown to illustrate the character of what was seen in the results?}
	
	\cck{Missing discussion of the results in Figure 14 with respect to the robustness of our conclusions. What things are reliably the same, (practically) regardless of the inversion parameters?}
	
	\cck{Mention something about computing performance. I think where we're at is that the cost is almost trivial--i.e., anybody can do this, provided they are paitient enough to wait a few hours. Also, do we want to make the code available--either by archiving current versions as supplementary material, or by pointing to a git repo?}
	  

